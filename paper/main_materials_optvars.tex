%
%
%


\subsubsection{Optimization Variables}
\label{sec:dimsynth_optvars}


The optimization variables in the dimensional synthesis affect the geometry of the parallel robot, shown exemplarily in Figure~\ref{fig:def_structural_entities}. Only parallel robots with identical leg chains are considered with coupling joints on the fixed base and moving platform that are set symmetrically on a circle circumference with radius $r_\mathrm{b}$.
The parameters for the base-coupling joints are summarized in Table~\ref{tab:basejointparams}.
The platform-coupling joints follow the same order with parameters using the index ``p'' instead of ``b.''
%
%
%
The direction of the coupling joint's axis of rotation or translation is selected according to geometric principles, which are also commonly found in existing robot structures in the literature.
The main principles are shown in Figure~\ref{fig:coupling_base_fig1} as vertical (v), tangential (t), radial (r), or conical (c) with respect to the base plane.
%
%
\nocite{FrindtKreHes2010} %
\nocite{SchuetzBudRaaHes2011} %
\nocite{HuangLiDin2012} %
\nocite{StechertFra2007} %


\begin{figure}[H] %
  \graphicspath{{./Figures/}}
  \input{./Figures/def_structural_entities.pdf_tex}
  \caption{Parallel robot with annotation of geometric structural entities. Mod. from \cite{FrindtKreHes2010}.}
  \label{fig:def_structural_entities}
\end{figure} 

%
%
%
%

\vspace{-9pt}
\begin{table}[H] %
  \centering
  %
  \caption{Dimensional parameters for the base-joint alignments in this paper.}
  \label{tab:basejointparams}
  \begin{tabularx}{\textwidth}{cCCCC|CCCC} %
    %
    %
    \toprule
    {\textbf{Position}} & \multicolumn{4}{c|}{\textbf{Symmetric on Circle Circumference}} & \multicolumn{4}{c}{\textbf{Pairwise}} \\
    \midrule
    \hl{\textbf{Direction}} %
    %
    & v & t & r & c & V & T & R & C \\
    \midrule
    \textbf{Parameters} & $r_{\mathrm{b}}$ & $r_{\mathrm{b}}$ & $r_{\mathrm{b}}$ & $r_{\mathrm{b}}$, $\gamma_{\mathrm{b}}$ & $r_{\mathrm{b}}$, $d_{\mathrm{b}}$ & $r_{\mathrm{b}}$, $d_{\mathrm{b}}$ & $r_{\mathrm{b}}$, $d_{\mathrm{b}}$ & $r_{\mathrm{b}}$, $d_{\mathrm{b}}$, $\gamma_{\mathrm{b}}$  \\
    \midrule
    \textbf{Figure} %
    %
    & \ref{fig:coupling_base_fig1}a & \ref{fig:coupling_base_fig1}b & \ref{fig:coupling_base_fig1}c & \ref{fig:coupling_base_fig1}d & \ref{fig:coupling_base_fig2}a & \ref{fig:coupling_base_fig2}b & \ref{fig:coupling_base_fig2}c & \ref{fig:coupling_base_fig2}d \\
    \bottomrule
  \end{tabularx}
\end{table}%

Six-DoF robots are usually built with a \emph{pairwise alignment} of the base-coupling joint, as shown in Figure~\ref{fig:coupling_base_fig2}.
%
In this case, the coupling\hl{-}joint pair centers are aligned on a regular triangle, and the joint pairs are set around the angle bisector with a given distance $d_\mathrm{b}$.
Likewise, a circle with a radius $r_\mathrm{b}$ can be used for the joint center definition \cite{SchuetzBudRaaHes2011},
cf. \mbox{Figure~\ref{fig:coupling_base_fig2}c.}
Alternatively, a geometric description with two parameters by the joint positions' radius and an angle offset is possible, cf.~\cite{SuDuaZhe2001} and \cite{Daake2012} %
%
(p.\,70).
A distinction of a pairwise alignment to the previous modes is noted by capital letters, i.e., V for vertical, R for radial, T for tangential, and C for conical.
The axes of the joint pairs are parallel (oriented at the regular triangle).
Therefore, the conical alignment ``C'' geometrically corresponds to a pyramid.
%

\vspace{-6pt}
\begin{figure}[H]
  %
  \begin{adjustwidth}{-\extralength}{0cm}
    \centering
    \graphicspath{{Figures/}}
    \input{./Figures/coupling_base_fig1.pdf_tex}
  \end{adjustwidth}
  \caption{Geometric principles for circular alignments of the base-coupling joint with given coupling joint frame (the blue $z$-axis corresponds to first joint axis): (\textbf{a}) \underline{v}ertical %
    %
    (mod. from [Figure 9.10g] in~\cite{KongGos2007}), (\textbf{b}) \underline{t}angential (mod. from [Figure 9.14] in~\cite{HuangLiDin2012}), (\textbf{c}) \underline{r}adial (mod. from \cite{StechertFra2007}), (\textbf{d}) \underline{c}onical ([Figure~9.10f] in~\cite{KongGos2007}).} %
\label{fig:coupling_base_fig1}
\end{figure}

\vspace{-9pt}
\begin{figure}[H]
\begin{adjustwidth}{-\extralength}{0cm}
  \centering
  \graphicspath{{Figures/}}
  \input{./Figures/coupling_base_fig2.pdf_tex}
\end{adjustwidth}
\caption{Geometric principles for pairwise circular alignments of the base-coupling joint with given coupling joint frames: (\textbf{a}) \underline{V}ertical (mod. from [Figure 2f] in \cite{FrindtKreHes2010}), (\textbf{b}) \underline{T}angential (mod. from [Figure~2a] in  \cite{FrindtKreHes2010}), (\textbf{c}) \underline{R}adial (top view on base), (\textbf{d}) \underline{C}onical/pyramidal (source: Daniel Ramirez, LUH;~mod).}%
\label{fig:coupling_base_fig2}
\end{figure}


%
The parameter in the optimization parameters $\bm{p}$ without direct correspondence in the physical kinematic parameters $\bm{p}_\mathrm{kin}$ is
\begin{enumerate}
\item \label{itm:param_scale}  the scaling parameter $p_\mathrm{scale}$, which increases the size of the whole robot (platform and legs) without changing its overall kinematics characteristics.%
\end{enumerate}
%
%

The %
%
optimization parameters corresponding to the \emph{base and platform} are
\begin{enumerate}
\setcounter{enumi}{1}
\item \label{itm:param_basescale} the base scaling parameter $p_\mathrm{b}$, giving the base radius by $r_\mathrm{b}=p_\mathrm{scale} p_\mathrm{b}$,
\item \label{itm:param_plfscale} the platform scaling parameter $p_\mathrm{plfscale}$ that sets the moving-platform radius relative to the base by $r_\mathrm{p}=p_\mathrm{plfscale} r_\mathrm{b}$,
\item \label{itm:param_basepairdist} the base-coupling\hl{-}joint pair-distance scaling parameter $p_\mathrm{bpdscale}$, giving the distance\\ $d_\mathrm{b}=p_\mathrm{bpdscale} r_\mathrm{b}$ of the joint pairs, in case of the alignments V, T, R, or C, %
\item \label{itm:param_platformpairdist} the similar pair-distance scaling parameter $p_\mathrm{ppdscale}$ with $d_\mathrm{p}=r_\mathrm{p} p_\mathrm{ppdscale}$ for the moving platform for platform-coupling\hl{-}joint alignments V, T, and R,
\item \label{itm:param_baseangle} the elevation $\gamma_\mathrm{b}$ of the base-coupling joint, in case of a conical or pyramidal alignment of the first joint axis of the leg chains (alignments c and C), and
\item \label{itm:param_platformangle} the angle inclination $\gamma_\mathrm{p}$ for platform-coupling\hl{-}joint alignment c. %
\end{enumerate}
%
The %
%
angular parameters \ref*{itm:param_baseangle} and \ref*{itm:param_platformangle} %
%
are optimized without scaling since their range of values is limited within \SI{\pm 90}{\degree}.
If limits are given for the base or platform radius, the scaling of parameters \ref*{itm:param_basescale} or \ref*{itm:param_plfscale} is omitted, and the physical values are optimized directly as $r_\mathrm{b}=p_\mathrm{b}$ or $r_\mathrm{p}=p_\mathrm{p}$ with a new parameter $p_\mathrm{p}$ instead of $p_\mathrm{plfscale}$.

For the \emph{leg chains}, the \emph{kinematic parameters} to optimize are $a_{i,\mathrm{DH}}$, $d_{i,\mathrm{DH}}$, $\alpha_{i,\mathrm{DH}}$, and $\theta_{i,\mathrm{DH}}$, in the modified \propername{Denavit}--\propername{Hartenberg} (DH) notation from Khalil \cite{KhalilDom2002}. %
Regarding optimization holds:
\begin{enumerate}
\setcounter{enumi}{7}
\item \label{itm:param_leglength} The leg-length parameters are scaled  by $a_{i,\mathrm{DH}}=p_\mathrm{scale} p_{a_i}$ and $d_{i,\mathrm{DH}}=p_\mathrm{scale} p_{d_i}$, and  %
\item \label{itm:param_legangle} the leg-angle parameters are optimized directly by $\alpha_{i,\mathrm{DH}}=p_{\alpha_i}$ and $\theta_{i,\mathrm{DH}}=p_{\theta_i}$.
\end{enumerate}
%
The %
%
number of kinematic parameters $\bm{p}_\mathrm{DH}=[p_{a_1},p_{d_1},p_{\alpha_1},p_{\theta_1},\dots,p_{a_n},p_{d_n},p_{\alpha_n},p_{\theta_n}]^\transp$ varies depending on the leg chain (with $n$ joints). %
%
The parameters $a_{1,\mathrm{DH}}$ and $d_{1,\mathrm{DH}}$ are removed from the optimization and set to zero since they are redundant with the base size or base $z$ position $r_{\mathrm{b},z}$, depending on the joint alignment.
This also holds for the $a_{2,\mathrm{DH}}$ and $d_{2,\mathrm{DH}}$ parameters if the first joint is prismatic.
If a prismatic joint $i$ is within the leg chain, a lift cylinder is assumed for technical realization (see constraint~\ref*{itm:constr_prismaticcylinder}) with a direct connection between the previous and the next joint, without a lever arm, resulting in $a_{i-1,\mathrm{DH}}=a_{i+1,\mathrm{DH}}=d_{i+1,\mathrm{DH}}=0$. %
%
For some parameters $\alpha_{i,\mathrm{DH}}$ or $\theta_{i,\mathrm{DH}}$, only constant values of \SI{0}{\degree} or \SI{90}{\degree} can be set as a result of the structural synthesis. %
Otherwise, angles are optimized continuously from \SI{-90}{\degree} to \SI{+90}{\degree}.
To be able to reproduce results with the parameter models from the literature, perpendicular connections between the joints (by $d_{i,\mathrm{DH}}=0$) and without skew joint-ax\hl{e}s alignments (by rounding $p_{\alpha_i}$ to  $\alpha_{i,\mathrm{DH}} \in \{0, \pm{}\SI{90}{\degree}\}$) can be enforced. %

The \emph{relative position and orientation of robot} (base frame $\ks{\indks{0}}$) and task within the world frame $\ks{\indks{W}}$ are optimized by
\begin{enumerate}
\setcounter{enumi}{9}
\item \label{itm:param_basepos} the base position $\rovec{\indks{W}}{0}{=}[r_{\mathrm{b},x},r_{\mathrm{b},y},r_{\mathrm{b},z}]^\transp$ and
\item \label{itm:param_baseori} the base orientation, expressed as $\rotmat{\indks{W}}{0}{=}\bm{R}([\varphi_{\mathrm{b},x},\varphi_{\mathrm{b},y},\varphi_{\mathrm{b},z}]^\transp)$ using a rotation matrix $\bm{R}$ obtained from intrinsic $X$-$Y'$-$Z''$ \propername{Euler} angles.
\end{enumerate}
Depending on the robot's DoFs and the settings, some parameters remain constant, e.g., the base tilting angles $\varphi_{\mathrm{b},y}$ and $\varphi_{\mathrm{b},z}$ in the case of a translational 3T0R parallel robot.
Similarly, an \emph{additional end-effector transformation} between the platform frame $\ks{\indks{P}}$ (located in the middle of the moving platform) and the end-effector frame  $\ks{\indks{E}}$ with the tool center point as origin is optimized by
\begin{enumerate}
\setcounter{enumi}{11}
\item \label{itm:param_eepos} the end-effector position $\rovec{\indks{P}}{\indks{E}}{=}[r_{\mathrm{p},x},r_{\mathrm{p},y},r_{\mathrm{p},z}]^\transp$ relative to the platform frame and
\item \label{itm:param_eeori} the end-effector orientation expressed as $\rotmat{\indks{P}}{\indks{E}}{=}\bm{R}([\varphi_{\mathrm{p},x},\varphi_{\mathrm{p},y},\varphi_{\mathrm{p},z}]^\transp)$.
\end{enumerate}
%
Again, %
%
some of these parameters can remain constant or can be omitted. %
%
If an existing commercially available robot is optimized for a given task, only parameters \ref*{itm:param_basepos}--\ref*{itm:param_eeori} have to be optimized.
The optimization of a complex kinematic structure leads to a high number of more than ten parameters.
The total vector of (possible) optimization parameters is
\begin{equation}
\bm{p}=[p_\mathrm{scale}, p_\mathrm{b}, p_\mathrm{plfscale}, p_\mathrm{bpdscale}, p_\mathrm{ppdscale}, \gamma_\mathrm{b}, \gamma_\mathrm{p}, \bm{p}_\mathrm{DH}^\transp, r_{\mathrm{b},x}, \dots, \varphi_{\mathrm{b},z}, r_{\mathrm{p},x}, \dots, \varphi_{\mathrm{p},z}]^\transp.
\end{equation}
