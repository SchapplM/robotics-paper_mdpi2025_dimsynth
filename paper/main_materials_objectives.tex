%
%
%



\subsubsection{Objective Function}
\label{sec:ds_objective}



The objectives investigated primarily are
\begin{enumerate}
  \item \label{itm:obj_poserr} maximum position error (precision) of the tool center point, cf.~\cite{Merlet2006a},
  \item \label{itm:obj_maxactforce} maximum actuator force (in the case of the same type of actuators),
  \item \label{itm:obj_maxactvelo} maximum actuator velocity (similar to number~\ref*{itm:obj_maxactforce}),
  \item \label{itm:obj_power} maximum actuators' rated power as a product of maximum motor torque and speed,
  \item \label{itm:obj_energy} energy consumption during the reference trajectory (cf.~\cite{SchapplerOrt2020}),
  \item \label{itm:obj_stiffness} stiffness of the end-effector platform,
  \item \label{itm:obj_mass} mass of the structure assuming hollow cylinders and a solid platform plate. %
\end{enumerate}

Criteria %
%
\ref*{itm:obj_maxactforce} and \ref*{itm:obj_maxactvelo} can be used directly for dimensioning drives since, in combination, they present a speed-torque diagram.
If different structures with prismatic and revolute actuation should be compared, the power criterion (no. \ref*{itm:obj_power}) may support finding motors with small dimensioning.
These criteria are advantageous if no automated drive-train optimization is performed. %
Criteria \ref*{itm:obj_poserr}, \ref*{itm:obj_energy}, \ref*{itm:obj_stiffness}, or \ref*{itm:obj_mass} should be used according to the task requirements.
The {maximum} trajectory value for criteria \ref*{itm:obj_poserr}, \ref*{itm:obj_maxactforce}, \ref*{itm:obj_maxactvelo}, or \ref*{itm:obj_power} is minimized in the optimization.
To maximize the mechanical stiffness with criterion~\ref*{itm:obj_stiffness}, the largest compliance eigenvalue (from the inverse translational stiffness matrix over the trajectory, by the model from \cite{Krefft2006} (p.\,49\,ff) is minimized.

Some criteria are designed as indirect measures for the ability to build up a working prototype, such as
\begin{enumerate}
  \setcounter{enumi}{7}
  \item \label{itm:obj_materialstress} material stress (internal forces in relation to the material's yield strength; see~\cite{SchapplerOrt2020}),
  \item \label{itm:obj_linklength} length of the links (summed over all leg chains),
  \item \label{itm:obj_colldist} maximization of the smallest collision distance during reference motion (cf. constraint~\ref{itm:constr_selfcoll}),
  \item \label{itm:obj_installspace} installation-space volume from a convex volume containing all joint positions throughout the trajectory, obtained by the alpha-shape algorithm \cite{MatlabAS},
  \item \label{itm:obj_footprint} footprint, i.e., the floor area taken by the robot projected from the points of the installation space, using the alpha-shape algorithm \cite{MatlabAS}, and
  \item \label{itm:obj_jointrange} used joint-coordinate range in relation to the maximum allowed joint range. If absolute position limits are given, they can also be used for the criterion.
\end{enumerate}

\newpage %
Most %
%
of these criteria, like \ref*{itm:obj_materialstress}, \ref*{itm:obj_colldist}, \ref*{itm:obj_installspace}, and \ref*{itm:obj_jointrange}, are also suited as constraints, but using them as objectives may shift the solution towards a desired direction in the design space.
When, e.g., minimizing the material stress (no.~\ref*{itm:obj_materialstress}), the link dimensioning will likely not pose a problem, and relatively low internal forces can be expected.
An optimization for short links (no.~\ref*{itm:obj_linklength}) is likely also to produce a compact solution, similar to the optimization for small installation-space volume (no.~\ref*{itm:obj_installspace}) or footprint area (no.~\ref*{itm:obj_footprint}).
Searching for the largest collision distance (no. \ref*{itm:obj_colldist}) will produce solutions that are more likely to be easily assembled, will be less dangerous in human--robot collaboration \cite{MohammadSeeSch2024}, and will not suffer from self-collision if parameters are changed.
Some of these objectives may also conflict, e.g., a large collision distance may produce a robot with a large installation space.

When performing a structural synthesis with the framework, the single objective is
\begin{enumerate}
  \setcounter{enumi}{13}
  \item \label{itm:obj_rankjacobian} the rank deficiency of the manipulator \propername{Jacobian}.
\end{enumerate}

Other objectives present more mathematical than physical values, such as
\begin{enumerate}
  \setcounter{enumi}{14}
  \item \label{itm:obj_condition} maximum condition number of the manipulator \propername{Jacobian},
  \item \label{itm:obj_manip} manipulability (based on the \propername{Jacobian}, \cite{Merlet2006a}),
  \item \label{itm:obj_singval} smallest singular value of the \propername{Jacobian}.
\end{enumerate}
%

%

%

The correlation of the \propername{Jacobian} condition number (objective~\ref*{itm:obj_condition}) with physical performance criteria (e.g., position-error objective~\ref*{itm:obj_poserr}) is questionable \cite{Merlet2006a}.
Manipulability (no.~\ref*{itm:obj_manip}) and single singular values (no.~\ref*{itm:obj_singval}) are criteria specific to every structure but are used for optimization within the synthesis or the redundancy resolution by some authors.
To summarize, comparing different robots or even robots with different types of actuation by their \propername{Jacobian} matrices is not feasible.
%
However, objectives \ref*{itm:obj_condition}--\ref*{itm:obj_singval} can still be used to produce sufficiently dimensioned academic examples or as a first iteration to explore the solution space for a dimensional synthesis.
Optimizing the average (``global'') condition number within the synthesis is not promising, as stated in \cite{JamwalHusXie2015}, since single near-singular poses have a strong influence.
In contrast, singular poses have no effect when optimizing the \emph{minimum} condition number, as in \cite{SuDuaZhe2001}.
A better alternative presents the use of the maximum condition number (objective~\ref*{itm:obj_condition}) within the given reference points or workspace.

Again, the specific implementation of the objectives is omitted at this point, and the reader is referred to the fundamental textbooks referenced above and the open-source implementation  %
%
\cite{GitHub_StructDimSynth} within the ydirectory \texttt{dimsynth/fitness} (files \texttt{cds\_obj\_...}).
